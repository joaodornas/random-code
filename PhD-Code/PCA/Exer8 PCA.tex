\documentclass[a4paper,12pt]{article}
\usepackage{graphicx}
\usepackage{amssymb}
\usepackage{amsmath}
\usepackage{epstopdf}
\usepackage[dvips]{epsfig}
\DeclareGraphicsRule{.tif}{png}{.png}{`convert #1 `basename #1 .tif`.png}

\newcommand{\be}{\begin{equation}}
\newcommand{\ee}{\end{equation}}
\newcommand{\ba}{\begin{eqnarray*}}
\newcommand{\ea}{\end{eqnarray*}}
\newcommand{\by}{\begin{array}}
\newcommand{\ey}{\end{array}}
\newcommand{\no}{\nonumber \\}
\newcommand{\expect}[1]{\langle #1 \rangle}
\newcommand{\erfc}{\rm erfc}
\newcommand{\bs}[1]{\boldsymbol #1 }
\newcommand{\wpp}{w_{_+}}
\newcommand{\wmm}{w_{_-}}

\def\slfrac#1#2{{\mathord{\mathchoice   % Bruch mit schraegem Bruchstrich
        {\kern.1em\raise.5ex\hbox{$\scriptstyle#1$}\kern-.1em
        /\kern-.15em\lower.25ex\hbox{$\scriptstyle#2$}}
        {\kern.1em\raise.5ex\hbox{$\scriptstyle#1$}\kern-.1em
        /\kern-.15em\lower.25ex\hbox{$\scriptstyle#2$}}
        {\kern.1em\raise.4ex\hbox{$\scriptscriptstyle#1$}\kern-.1em
        /\kern-.14em\lower.25ex\hbox{$\scriptscriptstyle#2$}}
        {\kern.1em\raise.2ex\hbox{$\scriptscriptstyle#1$}\kern-.1em
        /\kern-.1em\lower.25ex\hbox{$\scriptscriptstyle#2$}}}}}
        
%define local variable "markanswers"
\newif\ifmarkanswers %default value is "false"

%enable the following line to mark all correct answers
%\markanswerstrue



\textwidth = 16 cm
\textheight = 26 cm
\oddsidemargin = 0.0 cm
\evensidemargin = 0.0 cm
\topmargin = -1.5 cm
\headheight = 0.0 cm
\headsep = 0.5 cm
\parskip = 0.2 cm
\parindent = 0.0 cm



\title{\bf Theoretical Neuroscience II\\Exercise 8\\ Principal Component Analysis (PCA)}
\author{Due date: Thursday, 9 July 2015}


\begin{document}
\maketitle

\section{Motivation}

Principal component analysis is an extremely useful tool for analyzing high-dimensional data.  It identifies correlated sets of variables, so called `principal components' and thereby reduces the dimensionality of the data.  

In this exercise, you will analyze multi-unit activity from a network of spiking neurons.  In such networks, quiescent periods with comparatively little activity are interrupted spontaneously by transient periods of intense collective activity (`bursts').  Your task is to investage the origin of these bursts.  How does excitation spread through the network?  Is there a stereotypical propagation path?  Or does each burst start in a unique way?


\section{Principal components analysis}

Given $n$ observations of $m$ variables, each with zero mean, collected in a matrix $\bs X$
\ba
\bs X =\left(  \by{cccc} x_{11} & x_{12} \ldots & x_{1n} \\ \vdots & \ddots & \ddots & \vdots \\ x_{m1} & x_{m2} & \ldots & x_{mn} \ey \right)  \in \mathbb{R}^{m \times n}, \qquad\qquad \sum_j \, x_{ij} = 0, \,\, \forall i
\ea
and the covariance matrix
\ba
\bs C_{\bs X} = \frac{1}{n-1} \, \bs X \, \bs X^T \in \mathbb{R}^{m \times m}
\ea
we seek an orthonormal transformation $\bs P  \in \mathbb{R}^{m \times m}$ such that the transformed observations have diagonal covariance:
\ba
\bs Y = \bs P \, \bs X  \in \mathbb{R}^{m \times n}, \qquad\qquad \bs C_{\bs Y} = \frac{1}{n-1} \, \bs Y \, \bs Y^T = \left(  \by{cccc} y_{11} & 0 &\ldots & 0 \\ 0 & y_{22} & \ldots & \vdots \\ \vdots & \vdots & \ddots & \vdots \\ 0 & 0 & \ldots & y_{mm} \ey \right) 
\ea

From the `skinny' singular value decomposition $\bs X^T = \bs U \bs S \bs V^T$, we obtain the left singular vectors $\bs U \in  \mathbb{R}^{n \times n}$, the singular values  $\bs S \in  \mathbb{R}^{n \times m}$, and the right singular vectors $\bs V  \in \mathbb{R}^{m \times m}$.

The desired orthonormal transformation is
\ba
\bs P = \bs V^T  \in \mathbb{R}^{m \times m}
\ea 
with the {\bf rows} of $\bs P$ being the {\bf principal components}.  The variance captured by each component is revealed by diagonal matrix $\bs S$.  Specifically, element $s_{ii}$ (the $i^{th}$ diagonal element) represents the variance captured by the $i^{th}$ principal component.

\section{Average burst}

Multi-unit-activity is provided in a Matlab-file {\bf MUA\_b\_t\_g}.  After loading this file, you have the number of bursts {\bf Nb}, the number of time points {\bf Nt}, the number of neurons groups {\bf Ng}, the vector of time points {\bf ti}, and the 3d array {\bf MUA\_b\_t\_g} of size {\bf [Nb, Nt, Ng]}, which holds the collective spike rate of each group of neurons for different time points and for different bursts.

Begin by averaging the activity over bursts, to obtain a 2d array {\bf MUA\_g\_t} of size {\bf [Ng, Nt]}!  Note that the variables (neuron groups) do now range over {\it rows}, whereas the observations (time points) now range over columns, in agreement with the matrix $\bs X$ above.

Plot the activity of each group of neurons as a function of time!

Subtract the mean from each row and save the subtracted means for later!

Compute the covariance matrix and plot with Matlab function {\bf pcolor}!

Perform the singular value decomposition (don't forget to transpose the input!), using the Matlab command {\bf [U, S, V] = svd( X' )}!  Plot the diagonal values of {\bf S} with Matlab function {\bf bar}!   How many principal components capture significant variance?

\section{Transformed activity}

Transform the observed activity into the orthonormal space of principal components, beginning with the average over bursts!

Plot the activity of each principal component as a function of time!

Compute the covariance matrix of the transformed activity and plot with Matlab function {\bf pcolor}!  The result should be a diagonal matrix!

Plot the time-dependent activity of the first three principal components, using Matlab function {\bf plot3}! 

Zero the activity of all but the first three principal components and project back into the original space of neuron groups!  Plot this `denoised' activity of each group of neurons as a function of time!

\section{Individual bursts}

Transform the activity of individual bursts into the orthonormal space of principal components!

Plot the time-dependent activity of the first three principal components!  Superimpose all bursts in the same plot, by repeatedly using Matlab function{\bf plot3}! 




\end{document}

